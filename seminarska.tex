\documentclass[11pt,a4paper]{article}

\usepackage[utf8]{inputenc}
\usepackage[T1]{fontenc}
\usepackage{lmodern}
\usepackage{amsmath,amssymb,amsfonts}
\usepackage{amsthm}
\usepackage{graphicx}
\usepackage{hyperref}
\usepackage{geometry}
\usepackage[slovene]{babel}      % Slovenian language support
\geometry{a4paper, margin=1in}
\usepackage{titlesec}
\usepackage{enumitem}
\usepackage{caption}
\usepackage{cite}

% Define the definition environment
\newtheorem{definition}{Definicija}
\newtheorem{lemma}{Lema}
\newtheorem{theorem}{Izrek}
% Define custom environment based on 'proof'
% \newtheorem{proof}{Dokaz}




% % Title formatting
\titleformat{\section}{\large\bfseries}{\thesection.}{0.5em}{}
\titleformat{\subsection}{\normalsize\bfseries}{\thesubsection.}{0.5em}{}

% Metadata
\title{\textbf{Bernsteinovi bazni polinomi treh spremenljivk}}
\author{Nejc Jenko, Petja Murnik}
% \author{Petja Murnik}
\date{\today}

\begin{document}

% Title Page
\maketitle
\break
% \begin{abstract}
% This is the abstract of your article. Write a concise summary of the purpose, methodology, results, and conclusions here.
% \end{abstract}

\section{Uvod}

Pri predavanjih predmeta RPGO smo obravnavali Bernsteinove bazne polinome ene
ter dveh spremenljivk, ki smo jih uporabili pri različnih aplikacijah pri numerični matematiki.
V tem delu bomo predstavili Bernsteinove bazne polinome treh spremenljivk, ki so 
razširitev prej omenjenih. 

\section{Prostor polinomov treh spremenljivk} 

\begin{definition}\label{def_prostor}
    Naj bo $d \in \mathbb{N}_0$. Prostor polinomov treh spremenljivk stopnje $n$ je definiran kot
    \begin{equation}
        \mathbb{P}_d = \left\{ \sum_{0 \le i  + j + k \le d} a_{ijk} x^i y^j z^k  ; a_{ijk} \in \mathbb{R}\right\}.
    \end{equation}
\end{definition}

\begin{lemma}
    Naj bo $\mathbb{P}_d$ kot v definiciji \ref{def_prostor}.
    Potem je $\dim \mathbb{P}_d = \binom{d+3}{3} $. 
    Nadalje monomi $\left\{x^i y^j z^k \right\}_{0 \le i  + j + k \le d}$ tvorijo njegovo bazo.
\end{lemma}

Preden se lotimo dokaza leme, vpeljimo noticajo odvodov, 
ki nam ba prišla prav tako pri dokazu kot tudi 
v nadaljevanju. Naj bo $\text{D}^l_w$ operator parcialniega odvoda po
spremenljivki $w$ reda $l$. Potem definiramo 
$\text{D}^\alpha := \text{D}^{\alpha_1}_x \text{D}^{\alpha_2}_y \text{D}^{\alpha_3}_z$
za $\alpha = \left(\alpha_1 , \alpha_2 , \alpha_3 \right)$,
kjer je $\alpha_1,\alpha_2, \alpha_3 \in \mathbb{N}_0$ 
ter vpeljimo še $|\alpha| := \alpha_1 + \alpha_2 + \alpha_3$.

\begin{proof}
    Opazimo najprej, da monomi oblike $\{x^i y^j z^k\}_{0 \le i  + j + k \le d}$
    razpenjanjo $\mathbb{P}_d$, kar sledi direktno it definicije 
    prostora. Opazimo tudi, da 
    $|\left(i, j , k\right): 0\le i+j+k \le d , i,j,k \in \mathbb{N}_0| = \binom{d+3}{3} = \dim\mathbb{P}_d$.
    Pokazati moramo še linearno neodvisnost.  

    Predpostavimo, da je polinom $p(x,y,z) =\sum_{0 \le i  + j + k \le d} a_{ijk} x^i y^j z^k $
    identično enak 0. Potem so tudi vsi njegovi mešani odvodi identično enaki 0.
    Po drugi strani pa z direktnim odvajanjem dobimo
    $D^i_xD^j_yD^z_k p(x,y,z)|_{x =0 ,y =0, z = 0} = a_{ijk}$ za vsak 
    $0 \le i  + j + k \le d$. Torej linearna neodvisnost velja, ter
    je s tem dokazana lema.
\end{proof}

V nadaljevanju bomo prestavili drugo bazo prostora $\mathbb{P}_d$, zato 
nadaljne lastnosti baze monomov ne bomo raziskovali. Za uvedbo Bernsteinovih
baznih polinomov, potrebujemo uvedbo baricentričnih koordinat glede na 
tetraeder, kar je vpeljano v naslednjem razdelku.

Pred tem pa pa prestavimo še lemo, ki nam pove, kako oceniti 
normo polinoma $p \in \mathbb{P}_n$.

\begin{lemma}\label{lema_norma}
    Naj bo $T$ tetraeder z volumnom $V_T$, potem obstaja 
    konstanta $K$ odvisna le od $d$, da za vsak  $p \in \mathbb{P}_d$ ter $1 \le q < \infty$
    \begin{equation}
        V_T^{-1/q} \left\lVert p \right\rVert_{q,T} \le K \left\lVert p \right\rVert_{\infty l, T} \le K V_T^{-1/q} \left\lVert p \right\rVert_{q,T},
    \end{equation}
    kjer je $\left\lVert \dot{} \right\rVert_{q,T}$ standardna $L_q$ norma 
    glede na tetraeder $T$.
\end{lemma}

\begin{proof}
    TODO
\end{proof}

TODO 

\begin{theorem}
    TODO
\end{theorem}

\section{Baricentrične koordinate}

Analogno z ravninskim primerom, želimo tudi 
v $\mathbb{R}^3$ uvesti stabilnejšo bazo za prostor 
$\mathbb{P}_d$, ki bo temeljila na Bernsteinovih baznih polinomih,
za kar moramo najeprej vpeljati baricentrične koordinate glede na
tetraeder.

\begin{definition}
    Naj bo $T := \langle V_1 , V_2 , V_3, V_4\rangle$ nedegeneriran tetraeder
    v $\mathbb{R}^3$. To pomeni, da ima neničelen volumen. Rečemo, da so 
    vozlišča $V_1 , V_2 , V_3, V_4$ tetraedra $T$ v kanoničnem redu,
    če rotiramo in transliramo $T$ tako, da ploskev
    $\langle V_1, V_2, V_3\rangle$ leži v ravnini $z = 0$, je pozitivno
    orientiran ter je $z$ koordinata vozlišča $V_4$ pozitivna.
\end{definition}

Od sedaj naprej predpostavimo, da je tetraeder $T = \langle V_1, V_2, V_3,V_4 \rangle$ v kanoničnem redu,
kjer za vozlišče $V_i$ velja $V_i = (x_i, y_i, z_i)$. Postavimo 

\begin{align}
    M := \begin{bmatrix}
        1 & 1 & 1 & 1 \\
        x_1 & x_2 & x_3 & x_4 \\
        y_1 & y_2 & y_3 & y_4 \\
        z_1 & z_2 & z_3 & z_4
    \end{bmatrix},
\end{align}
kjer je znano dejstvo, da velja $\text{det}(M) = 6V_T$, kjer je $V_T$ volumen tetraedra $T$. 


\begin{lemma}\label{lema_baricentricne}
    Naj bo $T = \langle V_1, V_2, V_3, V_4 \rangle$ tetraeder v kanoničnem redu.
    Potem za vsako točko $V = (x,y,z) \in \mathbb{R}^3$ obstajajo enolično določene
     $\phi_1, \phi_2, \phi_3, \phi_4 \in \mathbb{R}$,
    da velja 
    \begin{equation}\label{eq_baricentricne}
        V = \phi_1 V_1 + \phi_2 V_2 + \phi_3 V_3 + \phi_4 V_4
    \end{equation} ter 
    \begin{equation}\label{eq_particija}
        \phi_1 + \phi_2 + \phi_3 + \phi_4 = 1 .
    \end{equation}
    Količinam $\phi_1, \ldots, \phi_4$ rečemo baricentrične 
    koordinate točke $V$ glede na tetraeder $T$.
\end{lemma}

\begin{proof}
    Enačbi \ref{eq_baricentricne} in \ref{eq_particija} sta ekvivalentni
    sledečemu nesingularnem sistemu:
    \begin{align}
        M \begin{bmatrix}\phi_1 \\ \phi_2 \\ \phi_3 \\ \phi_4 \end{bmatrix} = \begin{bmatrix}
            1 \\ x \\ y \\ z \end{bmatrix}.
    \end{align}
\end{proof}

V dokazu opazimo, da z uporabo Cramerjevega pravila dobimo
\begin{align}\label{eq_cramer}
    \phi_1 = \frac{1}{\text{det}(M)}
    \begin{vmatrix}
        1 & 1 & 1 & 1 \\
        x & x_2 & x_3 & x_4 \\
        y & y_2 & y_3 & y_4 \\
        z & z_2 & z_3 & z_4
    \end{vmatrix}.
\end{align}
Podobno bi dobili za $\phi_2, \phi_3, \phi_4$. 
Opazimo tudi, da lahko $i$-to baricentrično koordinato izračunamo
kot $\frac{V_{\widetilde{T_i}}}{V_T}$, kjer je $V_{\widetilde{T_i}}$
prostornina tetraedra, ki ga dobimo, če zamenjamo $i$-to vozlišče
s točko $V$. 
Naprej bi z razvojem determinante v izrazu \eqref{eq_cramer}
dokazali naslednjo lemo.

\begin{lemma}
    Za vsak $i = 1, 2, 3, 4$ je funkcija $\phi_i$ linearni polinom
    v $x, y, z$, ki doseže vrednost $1$ v oglišču $V_i$ in
    izgine v vseh točkah na ploskvi tetraedra $T$, ki leži
    nasproti $V_i$. Poleg tega velja
    $0 \leq \phi_i \leq 1$, kadar $(x, y, z)$ leži v $T$.
\end{lemma}

\section{Bernsteinovi bazni polinomi}

\begin{definition}
    Naj bo $T = \langle V_1, V_2, V_3, V_4 \rangle$ tetraeder v kanoničnem redu.
    Bernsteinov bazni polinom stopnje $n$ glede na tetraeder $T$ je definiran kot
    \begin{equation}
        B_{ijkl}^d := \frac{d!}{i!j!k!l!} \phi_1^i \phi_2^j \phi_3^k \phi_4^l, \quad i + j + k + l = d,
    \end{equation}
    kjer je $\phi_1, \phi_2, \phi_3, \phi_4$ baricentrične koordinate funkcije
    glede na tetraeder $T$. V primeru, ko je kateri izmed indeksov $i, j, k, l$ negativen, 
    nastavimo $B_{ijkl}^d = 0$.
\end{definition}

\begin{theorem}\label{izrek_bernstein}
    Bernsteinovi bazni polinomi $B_{ijkl}^d$ tvorijo bazo prostora $\mathbb{P}_d$.
    Prav tako velja 
    \begin{equation}
        \sum_{i+j+k+l = d} B_{ijkl}^d(V) = 1, \quad \forall V \in \mathbb{R}^3     
    \end{equation}
    ter
    \begin{equation}
        0 \leq B_{ijkl}^d(V) \leq 1, \quad \forall V \in T.
    \end{equation}
\end{theorem}

\begin{proof}
    TODO
\end{proof}

Iz izreka \ref{izrek_bernstein} sledi, da lahko vsak polinom $p \in \mathbb{P}_d$
lahko zapišemo na enoličen način kot
\begin{equation}\label{eq_Bforma}
    p = \sum_{i+j+k+l = d} c_{ijkl} B_{ijkl}^d.
\end{equation}
Tak zapis imenujemo \textbf{B-forma} ter koeficente $c_{ijkl}$ \textbf{B-koeficienti}.
Množico domenskih točk definiramo ter označimo kot
\begin{equation}
    \mathcal{D}_{d,T} := 
    \left\{
        \zeta_{ijkl}^T:= \frac{i V_1 + j V_2 + k V_3 + l V_4}{d}
     \right\}_{i + j+ k+l = d}.
\end{equation}
TODO slika z domenskimi točkami.

Nadalje pripišemo vsaki domenski točki $\zeta_{ijkl}^T$ B-koeficent
$c_{ijkl}$ za $i +j+k+l = d$. Torej lahko zapišemo B-koeficente kot $\left\{
    c_{\zeta}
\right\}_{\zeta \in \mathcal{D}_{d,T}}$ tako da če $\zeta = \zeta_{ijkl}^T$,
potem je $c_{\zeta} = c_{ijkl}$.

\begin{theorem}
    Naj bo $p \in \mathbb{P}_d$ zapisan v obliki \eqref{eq_Bforma}. Označimo 
    z $\text{F}_1 := \langle V2,V_3,V_4 \rangle$ ploskev tetraedra $T$ nasproti vozlišču $V_1$.
    Potem velja 
    \begin{equation}
        p|_{\text{F}_1} = \sum_{j+k+l = d} c_{0jkl}B_{0jkl}^d = 
            \sum_{j+k+l}c_{jkl}^{\text{F}_1} B_{jkl}^{\text{F}_1,d},
    \end{equation}
    kjer $c_{jkl}^{\text{F}_1}:=c_{0jkl}$ ter so $B_{jkl}^{\text{F}_1,d}$
    Bernsteinovi bazni polinomi stopnje $d$ glede na trikotnik $\text{F}_1$ .
\end{theorem}

Podobne formule veljajo za ostale ploskve tetraedra $T$.

\begin{proof}
    Upoštevaje dejstvo, da je $B_{0jkl}^d = 0$ za vsako $V \in \text{F}_1$ (ker je 
    tam $\phi_1 = 0$), trditev sledi nemudoma.
\end{proof}

\section{de Castljaujev algoritem}


\end{document}

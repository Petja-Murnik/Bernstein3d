\documentclass[11pt,a4paper]{article}

\usepackage[utf8]{inputenc}
\usepackage[T1]{fontenc}
\usepackage{lmodern}
\usepackage{amsmath,amssymb,amsfonts}
\usepackage{amsthm}
\usepackage{graphicx}
\usepackage{hyperref}
\usepackage{geometry}
\usepackage[slovene]{babel}      % Slovenian language support
\geometry{a4paper, margin=1in}
\usepackage{titlesec}
\usepackage{enumitem}
\usepackage{caption}
\usepackage{cite}

% Define the definition environment
\newtheorem{definition}{Definicija}
\newtheorem{lemma}{Lema}
% Define custom environment based on 'proof'
% \newtheorem{proof}{Dokaz}




% % Title formatting
\titleformat{\section}{\large\bfseries}{\thesection.}{0.5em}{}
\titleformat{\subsection}{\normalsize\bfseries}{\thesubsection.}{0.5em}{}

% Metadata
\title{\textbf{Bernsteinovi bazni polinomi treh spremenljivk}}
\author{Nejc Jenko, Petja Murnik}
% \author{Petja Murnik}
\date{\today}

\begin{document}

% Title Page
\maketitle
\break
% \begin{abstract}
% This is the abstract of your article. Write a concise summary of the purpose, methodology, results, and conclusions here.
% \end{abstract}

\section{Uvod}

Pri predavanjih predmeta RPGO smo obravnavali Bernsteinove bazne polinome ene
ter dveh spremenljivk, ki smo jih uporabili pri različnih aplikacijah pri numerični matematiki.
V tem delu bomo predstavili Bernsteinove bazne polinome treh spremenljivk, ki so 
razširitev prej omenjenih. 

\section{Prostor polinomov treh spremenljivk} 

\begin{definition}\label{def_prostor}
    Naj bo $n \in \mathbb{N}_0$. Prostor polinomov treh spremenljivk stopnje $n$ je definiran kot
    \begin{equation}
        \mathbb{P}_n = \left\{ \sum_{0 \le i  + j + k \le n} a_{ijk} x^i y^j z^k  ; a_{ijk} \in \mathbb{R}\right\}.
    \end{equation}
\end{definition}

\begin{lemma}
    Naj bo $\mathbb{P}_n$ kot v definiciji \ref{def_prostor}.
    Potem je $\dim \mathbb{P}_n = \binom{n+3}{3} $. 
    Nadalje monomi $\left\{x^i y^j z^k \right\}_{0 \le i  + j + k \le n}$ tvorijo njegovo bazo.
\end{lemma}

Preden se lotimo dokaza leme, vpeljimo noticajo odvodov, 
ki nam ba prišla prav tako pri dokazu kot tudi 
v nadaljevanju. Naj bo $\text{D}^l_w$ operator parcialniega odvoda po
spremenljivki $w$ reda $l$. Potem definiramo 
$\text{D}^\alpha := \text{D}^{\alpha_1}_x \text{D}^{\alpha_2}_y \text{D}^{\alpha_3}_z$
za $\alpha = \left(\alpha_1 , \alpha_2 , \alpha_3 \right)$,
kjer je $\alpha_1,\alpha_2, \alpha_3 \in \mathbb{N}_0$ 
ter vpeljimo še $|\alpha| := \alpha_1 + \alpha_2 + \alpha_3$.

\begin{proof}
    Opazimo najprej, da monomi oblike $\{x^i y^j z^k\}_{0 \le i  + j + k \le n}$
    razpenjanjo $\mathbb{P}_n$, kar sledi direktno it definicije 
    prostora. Opazimo tudi, da 
    $|\left(i, j , k\right): 0\le i+j+k \le n , i,j,k \in \mathbb{N}_0| = \binom{n+3}{3} = \dim\mathbb{P}_n$.
    Pokazati moramo še linearno neodvisnost.  

    Predpostavimo, da je polinom $p(x,y,z) =\sum_{0 \le i  + j + k \le n} a_{ijk} x^i y^j z^k $
    identično enak 0. Potem so tudi vsi njegovi mešani odvodi identično enaki 0.
    Poi drugi strani pa z direktnim odvajanjem dobimo
    $D^i_xD^j_yD^z_k p(x,y,z)|_{x =0 ,y =0, z = 0} = a_{ijk}$ za vsak 
    $0 \le i  + j + k \le n$. Torej linearna neodvisnost velja, ter
    je s tem dokazana lema.
\end{proof}



\section{Conclusion}
Summarize the key findings and suggest future work or implications.

% \section*{References}
% \bibliographystyle{plain}
% \bibliography{references}

$\left\lVert 3 \right\rVert$ 


\end{document}
